
\documentclass[a4paper,12pt]{article}

% Этот шаблон документа разработан в 2014 году
% Данилом Фёдоровых (danil@fedorovykh.ru) 
% для использования в курсе 
% <<Документы и презентации в \LaTeX>>, записанном НИУ ВШЭ
% для Coursera.org: http://coursera.org/course/latex .
% Исходная версия шаблона --- 
% https://www.writelatex.com/coursera/latex/5.3

% В этом документе преамбула

%%% Работа с русским языком
\usepackage{cmap}					% поиск в PDF
\usepackage{mathtext} 				% русские буквы в формулах
\usepackage[T2A]{fontenc}			% кодировка
\usepackage[utf8x]{inputenc}			% кодировка исходного текста
\usepackage[english,russian]{babel}	% локализация и переносы
\usepackage{indentfirst}
\frenchspacing

\renewcommand{\epsilon}{\ensuremath{\varepsilon}}
\renewcommand{\phi}{\ensuremath{\varphi}}
\renewcommand{\kappa}{\ensuremath{\varkappa}}
\renewcommand{\le}{\ensuremath{\leqslant}}
\renewcommand{\leq}{\ensuremath{\leqslant}}
\renewcommand{\ge}{\ensuremath{\geqslant}}
\renewcommand{\geq}{\ensuremath{\geqslant}}
\renewcommand{\emptyset}{\varnothing}

%%% Дополнительная работа с математикой
\usepackage{amsmath,amsfonts,amssymb,amsthm,mathtools} % AMS
\usepackage{icomma} % "Умная" запятая: $0,2$ --- число, $0, 2$ --- перечисление

%% Номера формул
%\mathtoolsset{showonlyrefs=true} % Показывать номера только у тех формул, на которые есть \eqref{} в тексте.
%\usepackage{leqno} % Нумереация формул слева

%% Свои команды
\DeclareMathOperator{\sgn}{\mathop{sgn}}
\DeclareMathOperator{\grad}{grad}
\DeclareMathOperator{\Div}{div}
\DeclareMathOperator{\rot}{rot}


%% Перенос знаков в формулах (по Львовскому)
\newcommand*{\hm}[1]{#1\nobreak\discretionary{}
{\hbox{$\mathsurround=0pt #1$}}{}}

%%% Работа с картинками
\usepackage{graphicx}  % Для вставки рисунков
\graphicspath{{figures/}}  % папки с картинками
\setlength\fboxsep{3pt} % Отступ рамки \fbox{} от рисунка
\setlength\fboxrule{1pt} % Толщина линий рамки \fbox{}
\usepackage{wrapfig} % Обтекание рисунков текстом

%%% Работа с таблицами
\usepackage{array,tabularx,tabulary,booktabs} % Дополнительная работа с таблицами
\usepackage{longtable}  % Длинные таблицы
\usepackage{multirow} % Слияние строк в таблице

%%% Теоремы
\theoremstyle{plain} % Это стиль по умолчанию, его можно не переопределять.
\newtheorem{theorem}{Теорема}
\newtheorem*{thm}{Теорема}
\newtheorem{proposition}[theorem]{Утверждение}
 
\theoremstyle{definition} % "Определение"
\newtheorem{corollary}{Следствие}[theorem]
\newtheorem*{dfn}{Определение}
\newtheorem{problem}{Задача}
\newtheorem*{problem*}{Задача}

 
\theoremstyle{remark} % "Примечание"
\newtheorem*{nonum}{Решение}

%%% Программирование
\usepackage{etoolbox} % логические операторы

%%% Страница
%\usepackage{extsizes} % Возможность сделать 14-й шрифт
%\usepackage{geometry} % Простой способ задавать поля
%	\geometry{top=25mm}
%	\geometry{bottom=35mm}
%	\geometry{left=35mm}
%	\geometry{right=20mm}
 
\usepackage{fancyhdr} % Колонтитулы
%	\pagestyle{fancy}
 %	\renewcommand{\headrulewidth}{0pt}  % Толщина линейки, отчеркивающей верхний колонтитул
	%\lfoot{Нижний левый}
	%\rfoot{Нижний правый}
	%\rhead{Верхний правый}
	%\chead{Верхний в центре}
	%\lhead{Верхний левый}
	%\cfoot{Нижний в центре} % По умолчанию здесь номер страницы

\usepackage{setspace} % Интерлиньяж
%\onehalfspacing % Интерлиньяж 1.5
%\doublespacing % Интерлиньяж 2
%\singlespacing % Интерлиньяж 1

\usepackage{lastpage} % Узнать, сколько всего страниц в документе.

\usepackage{soul} % Модификаторы начертания

\usepackage{hyperref}
\usepackage[usenames,dvipsnames,svgnames,table,rgb]{xcolor}
\hypersetup{				% Гиперссылки
    unicode=true,           % русские буквы в раздела PDF
    pdftitle={Заголовок},   % Заголовок
    pdfauthor={Автор},      % Автор
    pdfsubject={Тема},      % Тема
    pdfcreator={Создатель}, % Создатель
    pdfproducer={Производитель}, % Производитель
    pdfkeywords={keyword1} {key2} {key3}, % Ключевые слова
    colorlinks=true,       	% false: ссылки в рамках; true: цветные ссылки
    linkcolor=red,          % внутренние ссылки
    citecolor=black,        % на библиографию
    filecolor=magenta,      % на файлы
    urlcolor=cyan           % на URL
}

\usepackage{csquotes} % Еще инструменты для ссылок

%\usepackage[style=apa,maxcitenames=2,backend=biber,sorting=nty]{biblatex}

\usepackage{multicol} % Несколько колонок

\usepackage{tikz} % Работа с графикой
\usepackage{pgfplots}
\usepackage{pgfplotstable}
\usepackage{coloremoji}
\usepackage{floatrow}
\usepackage{subcaption}
\newcommand*{\N}{\mathbb{N}}
\newcommand*{\R}{\mathbb{R}}
\newcommand*{\K}{\mathbb{K}}
\newcommand*{\V}{\mathcal{V}}
\newcommand*{\A}{\mathcal{A}}
\newcommand*{\ii}{\mathbf{1}}
\newcommand*{\oo}{\mathbf{0}}
\newcommand*{\ba}{\mathbf{a}}
\newcommand*{\bb}{\mathbf{b}}
\newcommand*{\Q}{\mathbb{Q}}
\graphicspath{{images/}}
%\usepackage{breqn}

\renewcommand\thesubfigure{\asbuk{subfigure}}
%\addbibresource{master.bib}

\usepackage{import}
\usepackage{pdfpages}
\usepackage{transparent}
\usepackage{xcolor}
\usepackage{xifthen}

%\newcommand{\incfig}[1]{%
%    \def\svgwidth{\columnwidth}
%    \import{./figures/}{#1.pdf_tex}
%}


\newcommand{\incfig}[2][1]{%
    \def\svgwidth{#1\columnwidth}
    \import{./figures/}{#2.pdf_tex}
}
\usepackage{titlesec}
\titleformat{\section}{\normalfont\Large\bfseries}{}{0pt}{}


\pdfsuppresswarningpagegroup=1
\pgfplotsset{compat=1.16}
\renewcommand{\thesection}{}
\renewcommand{\thesubsection}{\arabic{subsection}}
%%\setcounter{secnumdepth}{0}
%%% Local Variables:
%%% mode: latex
%%% TeX-master: "master"
%%% End:


\title{Вопросы к экзамену по теорполу}
%\author{Драчов Ярослав\\ Факультет общей и прикладной физики, МФТИ}


\begin{document} % конец преамбулы, начало документа
%\input{titlepage}
%\pdfsuppresswarningpagegroup=1a
\maketitle
\tableofcontents
\section{}
\subsection{Преобразование Лоренца вдоль направления оси $x$. Релятивистские
$\gamma$- и $\beta$-факторы.}
\begin{dfn}
	Здесь может быть \emph{определение} чего-нибудь.
\end{dfn}
\begin{thm}[Кого-нибудь о чём-нибудь]
	Здесь может быть теорема.
\end{thm}
\subsection{Световой конус и относительность одновременности.}
hi
\subsection{Лоренцево сокращение длин.}
hi
\subsection{Релятивистское сложение скоростей вдоль одного и того же
направления.}
hi
\subsection{Вычислить $\operatorname{div} \vec{r}$}
hi
\subsection{Вектор 4-скорости и 4-ускорения и их скалярное произведение.}
hi
\subsection{Компоненты 4-импульса и связь энергии с трехмерным импульсом.}
hi
\subsection{Преобразование Лоренца произвольного вектора при бусте вдоль оси
$x$.}
hi
\subsection{Эффективная масса нескольких частиц.}
hi
\subsection{Калибровочные преобразования потенциалов.}
hi
\subsection{Калибровочные преобразования потенциалов в трехмерной форме.}
hi
\subsection{Калибровочные преобразования потенциалов в четырехмерной форме.}
hi
\subsection{Выражения для $E$ и $B$ через компоненты 4-потенциала.}
hi
\subsection{Выражение для тензора электромагнитного поля через 4-вектор
потенциал.}
hi
\subsection{Сила Лоренца.}
hi
\subsection{Скорость дрейфа в скрещенных электромагнитных полях.}
hi
\subsection{Магнитное зеркало.}
hi
\subsection{Вычислить среднее $\left<\left(\vec{a},\,\vec{n}\right)
\left( \vec{b},\,\vec{n} \right) \right>$ по всем направлениям единичного
вектора $\vec{n}$ при постоянных  $\vec{a}$, $\vec{b}$}
hi
\subsection{Вычислить среднее $\left<\left[\vec{a},\,\vec{n}\right]
\left( \vec{b},\,\vec{n} \right) \right>$ по всем направлениям единичного
вектора $\vec{n}$ при постоянных  $\vec{a}$, $\vec{b}$}
hi
\subsection{Четыре-вектор тока и его компоненты.}
hi
\subsection{Уравнение непрерывности в четырехмерной и трехмерной форме.}
hi
\subsection{Плотность энергии электромагнитного поля.}
hi
\subsection{Вектор Умова-Пойнтинга.}
hi
\subsection{Вектор потенциал А для плоской и монохроматической электромагнитной
волны.}
hi
\subsection{Поляризация плоской монохроматической электромагнитной волны.}
hi
\subsection{Векторы $E$, $B$ и Умова-Пойнтинга в плоской и монохроматической
электромагнитной волне.}
hi
\subsection{Классический радиус электрона и как он возникает в выражениях,
описывающих рассеяние электромагнитных волн.}
hi
\subsection{Аберрация света.}
hi
\subsection{Собственное время.}
hi
\subsection{Вычислить $\operatorname{grad} \frac{1}{\left| \vec{r} \right| }$}
hi
\subsection{Вычислить $\operatorname{grad}
	\frac{1}{\overrightarrow{\left( \vec{k},\,\vec{r}\right)}}
$, где $\vec{k}$ --- постоянный вектор.}
hi
\subsection{Вычислить $\operatorname{grad} e ^{i\left( \vec{k},\,\vec{r}
\right) }$, где $\vec{k}$ --- постоянный вектор.}
hi
\subsection{Вычислить $\epsilon_{ijk}x_i x_k$.}
hi
\subsection{Вычислить $\delta_{ij}\partial_i x_k$.}
hi
\subsection{Вычислить $\delta_{ij}\partial_i x_j$.}
hi
\subsection{Действие для свободной релятивистской частицы.}
hi
\subsection{Вывести формулу для эффекта Доплера.}
hi
\subsection{Может ли свободный электрон излучить фотон? Объяснение.}
hi
\subsection{Действие для релятивистской частицы во внешнем электромагнитном
поле.}
hi
\subsection{Уравнение движения для релятивистской частицы во внешнем
электромагнитном поле в 4-мерной форме.}
hi
\subsection{Обобщенный импульс и энергия.}
hi
\subsection{Получите инварианты поля в четырехмерной (через тензор поля)
исходя и тензора э-м поля.}
hi
\subsection{Инварианты электромагнитного поля в трехмерной форме (через $E$ и
$B$).}
hi
\subsection{Вычислить среднее $\left<\left[ \vec{a},\,\vec{r} \right] \vec{r}
\right>$ по всем направлениям вектора $\vec{r}$ при постоянных  $\left| 
\vec{r}\right| $, $\vec{a}$, $\vec{b}$.}
hi
\subsection{Вычислить среднее $\left<\left[ \vec{a},\,\vec{n} \right]
\left[ \vec{b},\,\vec{n} \right]
\right>$ по всем направлениям вектора $\vec{n}$ при постоянных 
$\vec{a}$, $\vec{b}$.}
hi
\subsection{Первая и вторая пара уравнений Максвелла в четырехмерной форме.}
hi
\subsection{Дипольный электрический момент и поле, создаваемое им.}
hi
\subsection{Квадрупольный момент.}
hi
\subsection{Энергия электрического диполя и квадруполя во внешнем поле.}
hi
\subsection{Потенциальная энергия взаимодействия диполя с диполем.}
hi
\subsection{Закон Био-Савара – магнитное поле, создаваемое стационарным током.}
hi
\subsection{Калибровка Лоренца и вторая пара уравнений Максвелла в ней.}
hi
\subsection{Калибровка Кулона и уравнение на три-вектор потенциал $A$ в
присутствии стационарного тока.}
hi
\subsection{Дипольный магнитный момент и поле, создаваемое им.}
hi
\subsection{Прецессия магнитного момента в магнитном поле. Частота Лармора.}
hi
\subsection{Запаздывающие потенциалы.}
hi
\subsection{Получить Потенциалы Лиенара-Вихерта в трехмерной и четырехмерной
форме из запаздывающих потенциалов.}
hi
\subsection{Волновая зона. Характер поведения полей $E$ и $B$ вблизи
двигающегося заряда.}
hi
\subsection{Длина формирования излучения или длина когерентности.}
hi
\subsection{Характер распределения по углам излучения в ультрарелятивистском
случае.}
hi
\subsection{Интенсивность излучения в дипольном приближении.}
hi
\subsection{Характерная частота при синхротронном излучении.}
hi
\subsection{Радиационная сила трения. Критерий применимости.}
hi
\subsection{Лоренцева линия. Естественная ширина линии.}
hi
\subsection{Формула Томсона для сечения рассеяния.}
hi
\subsection{Тензор электромагнитного поля и связь его компонент с $E$ и $B$.}
hi
\subsection{Гамильтониан частицы в нерелятивистском приближении во внешнем
электромагнитном поле.}
hi
\subsection{Можно ли превысить скорость света при движении под действием
постоянной силы? Объяснение.}
hi
\subsection{Четыре-вектор тока для точечной частицы.}
hi
\subsection{Первая и вторая пара уравнений Максвелла в трехмерной форме.}
hi
\subsection{Тензор энергии-импульса для точечной частицы.}
hi
\subsection{Тензор энергии-импульса электромагнитного поля.}
hi
\subsection{Закон сохранения тензора энергии-импульса.}
hi
\subsection{Уравнение Пуассона и его решение. Потенциал Кулона.}
hi
\subsection{Разложение электромагнитного поля на осцилляторы. Фурье разложение 
$A$, $E$ и $B$.}
hi
\subsection{Действие для осцилляторов (собственных колебаний)
электромагнитного поля.}
hi
\subsection{Запаздывающая функция Грина для электромагнитного поля и ее
свойства.}
hi
\subsection{Получить запаздывающие потенциалы из запаздывающей функции Грина.}
hi
\subsection{Характер зависимости поля произвольно двигающегося заряда от
расстояния. Сколько слагаемых в $E$ и $B$? Как они падают с расстоянием?
Как зависят от ускорения?}
hi
\subsection{Мощность потерь на излучение в релятивистском случае и его связь
с полной интенсивностью излучения.}
hi
\subsection{Критерий применимости силы радиацонного терия.}
hi
\subsection{Критерий применимости нерелятивистского приближения для излучения.}
hi
\subsection{Критерий применимости формулы Томсона для рассеяния.}
hi
\end{document} % конец документа
