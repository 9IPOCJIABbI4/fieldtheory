
\documentclass[a4paper,12pt]{article}

\input{preamble}

\title{Вопросы к экзамену по теорполу}
%\author{Драчов Ярослав\\ Факультет общей и прикладной физики, МФТИ}


\begin{document} % конец преамбулы, начало документа
%\input{titlepage}
%\pdfsuppresswarningpagegroup=1a
\maketitle
\tableofcontents
\section{}
\subsection{Преобразование Лоренца вдоль направления оси $x$. Релятивистские
$\gamma$- и $\beta$-факторы.}
\begin{dfn}
	Здесь может быть \emph{определение} чего-нибудь.
\end{dfn}
\begin{thm}[Кого-нибудь о чём-нибудь]
	Здесь может быть теорема.
\end{thm}
\subsection{Световой конус и относительность одновременности.}
hi
\subsection{Релятивистское сложение скоростей вдоль одного и того же
направления.}
hi
\subsection{Вычислить $\operatorname{div} \vec{r}$}
hi
\subsection{Вектор 4-скорости и 4-ускорения и их скалярное произведение.}
hi
\subsection{Компоненты 4-импульса и связь энергии с трехмерным импульсом.}
hi
\subsection{Преобразование Лоренца произвольного вектора при бусте вдоль оси
$x$.}
hi
\subsection{Эффективная масса нескольких частиц.}
hi
\subsection{Калибровочные преобразования потенциалов.}
hi
\subsection{Калибровочные преобразования потенциалов в трехмерной форме.}
hi
\subsection{Калибровочные преобразования потенциалов в четырехмерной форме.}
hi
\subsection{Выражения для $E$ и $B$ через компоненты 4-потенциала.}
hi
\subsection{Выражение для тензора электромагнитного поля через 4-вектор
потенциал.}
hi
\subsection{Сила Лоренца.}
hi
\subsection{Скорость дрейфа в скрещенных электромагнитных полях.}
hi
\subsection{Магнитное зеркало.}
hi
\subsection{Вычислить среднее $\left<\left(\vec{a},\,\vec{n}\right)
\left( \vec{b},\,\vec{n} \right) \right>$ по всем направлениям единичного
вектора $\vec{n}$ при постоянных  $\vec{a}$, $\vec{b}$}
hi
\subsection{Вычислить среднее $\left<\left[\vec{a},\,\vec{n}\right]
\left( \vec{b},\,\vec{n} \right) \right>$ по всем направлениям единичного
вектора $\vec{n}$ при постоянных  $\vec{a}$, $\vec{b}$}
hi
\subsection{Четыре-вектор тока и его компоненты.}
hi
\subsection{Уравнение непрерывности в четырехмерной и трехмерной форме.}
hi
\subsection{Плотность энергии электромагнитного поля.}
hi
\subsection{Вектор Умова-Пойнтинга.}
hi
\subsection{Вектор потенциал А для плоской и монохроматической электромагнитной
волны.}
hi
\subsection{Поляризация плоской монохроматической электромагнитной волны.}
hi
\subsection{Векторы $E$, $B$ и Умова-Пойнтинга в плоской и монохроматической
электромагнитной волне.}
hi
\subsection{Классический радиус электрона и как он возникает в выражениях,
описывающих рассеяние электромагнитных волн.}
hi
\subsection{Аберрация света.}
hi
\subsection{Собственное время.}
hi
\subsection{Вычислить $\operatorname{grad} \frac{1}{\left| \vec{r} \right| }$}
hi
\subsection{Вычислить $\operatorname{grad}
	\frac{1}{\overrightarrow{\left( \vec{k},\,\vec{r}\right)}}
$, где $\vec{k}$ --- постоянный вектор.}
hi
\subsection{Вычислить $\operatorname{grad} e ^{i\left( \vec{k},\,\vec{r}
\right) }$, где $\vec{k}$ --- постоянный вектор.}
hi
\subsection{Вычислить $\epsilon_{ijk}x_i x_k$.}
hi
\subsection{Вычислить $\delta_{ij}\partial_i x_k$.}
hi
\subsection{Вычислить $\delta_{ij}\partial_i x_j$.}
hi
\subsection{Действие для свободной релятивистской частицы.}
hi
\subsection{Вывести формулу для эффекта Доплера.}
hi
\subsection{Может ли свободный электрон излучить фотон? Объяснение.}
hi
\subsection{Действие для релятивистской частицы во внешнем электромагнитном
поле.}
hi
\subsection{равнение движения для релятивистской частицы во внешнем
электромагнитном поле в 4-мерной форме.}
hi
\subsection{Обобщенный импульс и энергия.}
hi
\subsection{Получите инварианты поля в четырехмерной (через тензор поля)
исходя и тензора э-м поля.}
hi
\subsection{Инварианты электромагнитного поля в трехмерной форме (через $E$ и
$B$).}
hi
\subsection{Вычислить среднее $\left<\left[ \vec{a},\,\vec{r} \right] \vec{r}
\right>$ по всем направлениям вектора $\vec{r}$ при постоянных  $\left| 
\vec{r}\right| $, $\vec{a}$, $\vec{b}$.}
hi
\subsection{Вычислить среднее $\left<\left[ \vec{a},\,\vec{n} \right]
\left[ \vec{b},\,\vec{n} \right]
\right>$ по всем направлениям вектора $\vec{n}$ при постоянных 
$\vec{a}$, $\vec{b}$.}
hi
\subsection{Первая и вторая пара уравнений Максвелла в четырехмерной форме.}
hi
\subsection{Дипольный электрический момент и поле, создаваемое им.}
hi
\subsection{Квадрупольный момент.}
hi
\subsection{Энергия электрического диполя и квадруполя во внешнем поле.}
hi
\subsection{Потенциальная энергия взаимодействия диполя с диполем.}
hi
\subsection{Закон Био-Савара – магнитное поле, создаваемое стационарным током.}
hi
\subsection{Калибровка Лоренца и вторая пара уравнений Максвелла в ней.}
hi
\subsection{Калибровка Кулона и уравнение на три-вектор потенциал $A$ в
присутствии стационарного тока.}
hi
\subsection{Дипольный магнитный момент и поле, создаваемое им.}
hi
\subsection{Прецессия магнитного момента в магнитном поле. Частота Лармора.}
hi
\subsection{Запаздывающие потенциалы.}
hi
\subsection{Получить Потенциалы Лиенара-Вихерта в трехмерной и четырехмерной
форме из запаздывающих потенциалов.}
hi
\subsection{Волновая зона. Характер поведения полей $E$ и $B$ вблизи
двигающегося заряда.}
hi
\subsection{Длина формирования излучения или длина когерентности.}
hi
\subsection{Характер распределения по углам излучения в ультрарелятивистском
случае.}
hi
\subsection{Интенсивность излучения в дипольном приближении.}
hi
\subsection{Характерная частота при синхротронном излучении.}
hi
\subsection{Радиационная сила трения. Критерий применимости.}
hi
\subsection{Лоренцева линия. Естественная ширина линии.}
hi
\subsection{Формула Томсона для сечения рассеяния.}
hi
\subsection{Тензор электромагнитного поля и связь его компонент с $E$ и $B$.}
hi
\subsection{Гамильтониан частицы в нерелятивистском приближении во внешнем
электромагнитном поле.}
hi
\subsection{Можно ли превысить скорость света при движении под действием
постоянной силы? Объяснение.}
hi
\subsection{Четыре-вектор тока для точечной частицы.}
hi
\subsection{Первая и вторая пара уравнений Максвелла в трехмерной форме.}
hi
\subsection{Тензор энергии-импульса для точечной частицы.}
hi
\subsection{Тензор энергии-импульса электромагнитного поля.}
hi
\subsection{Закон сохранения тензора энергии-импульса.}
hi
\subsection{Уравнение Пуассона и его решение. Потенциал Кулона.}
hi
\subsection{Разложение электромагнитного поля на осцилляторы. Фурье разложение 
$A$, $E$ и $B$.}
hi
\subsection{Действие для осцилляторов (собственных колебаний)
электромагнитного поля.}
hi
\subsection{Запаздывающая функция Грина для электромагнитного поля и ее
свойства.}
hi
\subsection{Получить запаздывающие потенциалы из запаздывающей функции Грина.}
hi
\subsection{Характер зависимости поля произвольно двигающегося заряда от
расстояния. Сколько слагаемых в $E$ и $B$? Как они падают с расстоянием?
Как зависят от ускорения?}
hi
\subsection{Мощность потерь на излучение в релятивистском случае и его связь
с полной интенсивностью излучения.}
hi
\subsection{Критерий применимости силы радиацонного терия.}
hi
\subsection{Критерий применимости нерелятивистского приближения для излучения.}
hi
\subsection{Критерий применимости формулы Томсона для рассеяния.}
hi
\end{document} % конец документа
