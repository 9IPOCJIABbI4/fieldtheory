
\documentclass[a4paper,12pt]{article}

% Этот шаблон документа разработан в 2014 году
% Данилом Фёдоровых (danil@fedorovykh.ru) 
% для использования в курсе 
% <<Документы и презентации в \LaTeX>>, записанном НИУ ВШЭ
% для Coursera.org: http://coursera.org/course/latex .
% Исходная версия шаблона --- 
% https://www.writelatex.com/coursera/latex/5.3

% В этом документе преамбула

%%% Работа с русским языком
\usepackage{cmap}					% поиск в PDF
\usepackage{mathtext} 				% русские буквы в формулах
\usepackage[T2A]{fontenc}			% кодировка
\usepackage[utf8x]{inputenc}			% кодировка исходного текста
\usepackage[english,russian]{babel}	% локализация и переносы
\usepackage{indentfirst}
\frenchspacing

\renewcommand{\epsilon}{\ensuremath{\varepsilon}}
\renewcommand{\phi}{\ensuremath{\varphi}}
\renewcommand{\kappa}{\ensuremath{\varkappa}}
\renewcommand{\le}{\ensuremath{\leqslant}}
\renewcommand{\leq}{\ensuremath{\leqslant}}
\renewcommand{\ge}{\ensuremath{\geqslant}}
\renewcommand{\geq}{\ensuremath{\geqslant}}
\renewcommand{\emptyset}{\varnothing}

%%% Дополнительная работа с математикой
\usepackage{amsmath,amsfonts,amssymb,amsthm,mathtools} % AMS
\usepackage{icomma} % "Умная" запятая: $0,2$ --- число, $0, 2$ --- перечисление

%% Номера формул
%\mathtoolsset{showonlyrefs=true} % Показывать номера только у тех формул, на которые есть \eqref{} в тексте.
%\usepackage{leqno} % Нумереация формул слева

%% Свои команды
\DeclareMathOperator{\sgn}{\mathop{sgn}}
\DeclareMathOperator{\grad}{grad}
\DeclareMathOperator{\Div}{div}
\DeclareMathOperator{\rot}{rot}


%% Перенос знаков в формулах (по Львовскому)
\newcommand*{\hm}[1]{#1\nobreak\discretionary{}
{\hbox{$\mathsurround=0pt #1$}}{}}

%%% Работа с картинками
\usepackage{graphicx}  % Для вставки рисунков
\graphicspath{{figures/}}  % папки с картинками
\setlength\fboxsep{3pt} % Отступ рамки \fbox{} от рисунка
\setlength\fboxrule{1pt} % Толщина линий рамки \fbox{}
\usepackage{wrapfig} % Обтекание рисунков текстом

%%% Работа с таблицами
\usepackage{array,tabularx,tabulary,booktabs} % Дополнительная работа с таблицами
\usepackage{longtable}  % Длинные таблицы
\usepackage{multirow} % Слияние строк в таблице

%%% Теоремы
\theoremstyle{plain} % Это стиль по умолчанию, его можно не переопределять.
\newtheorem{theorem}{Теорема}
\newtheorem*{thm}{Теорема}
\newtheorem{proposition}[theorem]{Утверждение}
 
\theoremstyle{definition} % "Определение"
\newtheorem{corollary}{Следствие}[theorem]
\newtheorem*{dfn}{Определение}
\newtheorem{problem}{Задача}
\newtheorem*{problem*}{Задача}

 
\theoremstyle{remark} % "Примечание"
\newtheorem*{nonum}{Решение}

%%% Программирование
\usepackage{etoolbox} % логические операторы

%%% Страница
%\usepackage{extsizes} % Возможность сделать 14-й шрифт
%\usepackage{geometry} % Простой способ задавать поля
%	\geometry{top=25mm}
%	\geometry{bottom=35mm}
%	\geometry{left=35mm}
%	\geometry{right=20mm}
 
\usepackage{fancyhdr} % Колонтитулы
%	\pagestyle{fancy}
 %	\renewcommand{\headrulewidth}{0pt}  % Толщина линейки, отчеркивающей верхний колонтитул
	%\lfoot{Нижний левый}
	%\rfoot{Нижний правый}
	%\rhead{Верхний правый}
	%\chead{Верхний в центре}
	%\lhead{Верхний левый}
	%\cfoot{Нижний в центре} % По умолчанию здесь номер страницы

\usepackage{setspace} % Интерлиньяж
%\onehalfspacing % Интерлиньяж 1.5
%\doublespacing % Интерлиньяж 2
%\singlespacing % Интерлиньяж 1

\usepackage{lastpage} % Узнать, сколько всего страниц в документе.

\usepackage{soul} % Модификаторы начертания

\usepackage{hyperref}
\usepackage[usenames,dvipsnames,svgnames,table,rgb]{xcolor}
\hypersetup{				% Гиперссылки
    unicode=true,           % русские буквы в раздела PDF
    pdftitle={Заголовок},   % Заголовок
    pdfauthor={Автор},      % Автор
    pdfsubject={Тема},      % Тема
    pdfcreator={Создатель}, % Создатель
    pdfproducer={Производитель}, % Производитель
    pdfkeywords={keyword1} {key2} {key3}, % Ключевые слова
    colorlinks=true,       	% false: ссылки в рамках; true: цветные ссылки
    linkcolor=red,          % внутренние ссылки
    citecolor=black,        % на библиографию
    filecolor=magenta,      % на файлы
    urlcolor=cyan           % на URL
}

\usepackage{csquotes} % Еще инструменты для ссылок

%\usepackage[style=apa,maxcitenames=2,backend=biber,sorting=nty]{biblatex}

\usepackage{multicol} % Несколько колонок

\usepackage{tikz} % Работа с графикой
\usepackage{pgfplots}
\usepackage{pgfplotstable}
\usepackage{coloremoji}
\usepackage{floatrow}
\usepackage{subcaption}
\newcommand*{\N}{\mathbb{N}}
\newcommand*{\R}{\mathbb{R}}
\newcommand*{\K}{\mathbb{K}}
\newcommand*{\V}{\mathcal{V}}
\newcommand*{\A}{\mathcal{A}}
\newcommand*{\ii}{\mathbf{1}}
\newcommand*{\oo}{\mathbf{0}}
\newcommand*{\ba}{\mathbf{a}}
\newcommand*{\bb}{\mathbf{b}}
\newcommand*{\Q}{\mathbb{Q}}
\graphicspath{{images/}}
%\usepackage{breqn}

\renewcommand\thesubfigure{\asbuk{subfigure}}
%\addbibresource{master.bib}

\usepackage{import}
\usepackage{pdfpages}
\usepackage{transparent}
\usepackage{xcolor}
\usepackage{xifthen}

%\newcommand{\incfig}[1]{%
%    \def\svgwidth{\columnwidth}
%    \import{./figures/}{#1.pdf_tex}
%}


\newcommand{\incfig}[2][1]{%
    \def\svgwidth{#1\columnwidth}
    \import{./figures/}{#2.pdf_tex}
}
\usepackage{titlesec}
\titleformat{\section}{\normalfont\Large\bfseries}{}{0pt}{}


\pdfsuppresswarningpagegroup=1
\pgfplotsset{compat=1.16}
\renewcommand{\thesection}{}
\renewcommand{\thesubsection}{\arabic{subsection}}
%%\setcounter{secnumdepth}{0}
%%% Local Variables:
%%% mode: latex
%%% TeX-master: "master"
%%% End:


\title{Вопросы к экзамену по теорполу}
%\author{Драчов Ярослав\\ Факультет общей и прикладной физики, МФТИ}


\begin{document} % конец преамбулы, начало документа
%\input{titlepage}
%\pdfsuppresswarningpagegroup=1a
\maketitle
\subsection*{Перед просмотром}
После названий некоторых вопросов указаны страницы с информацией по данной
теме в лекциях Э. Т. Ахмедова
\url{https://mipt.ru/upload/medialibrary/37f/lecturesto.pdf}.
\tableofcontents
\section{}
\begin{dfn}
	Здесь может быть \emph{определение} чего-нибудь.
\end{dfn}
\begin{thm}[Кого-нибудь о чём-нибудь]
	Здесь может быть теорема.
\end{thm}
В данном документе для 3-векторов используются латинские индексы, а для
4-векторов --- греческие.
\subsection{Преобразование Лоренца вдоль направления оси $x$. Релятивистские
$\gamma$- и $\beta$-факторы.}
\label{subsec:1}
Буст Лоренца со скоростью $v$ в положительном направлении оси  $x$
\[
	\begin{pmatrix} ct' \\ x' \\ y' \\ z' \end{pmatrix} =
	\begin{pmatrix} \gamma & -\beta \gamma & 0 & 0\\
	-\beta \gamma & \gamma & 0 & 0\\
0 & 0 & 1 & 0\\
0 & 0 & 0 & 1\end{pmatrix} 
\begin{pmatrix} ct \\ x \\ y \\ z \end{pmatrix} 
,\] 
где $\beta=v /c$, $\gamma = 1 / \sqrt{1-\beta^2} $.
\subsection{Световой конус и относительность одновременности \emph{(стр.~8)}}
\begin{dfn}
	Фигура, которая заметается в ПВ вссеми возможными лучами света,
	проходящими через данную точку, называется \emph{световой конус} 
	(см. рис.~\ref{fig:1}).
\end{dfn}
\begin{dfn}
	Точка в ПВ называется \emph{событием} или \emph{мировой точкой}.
\end{dfn}
\begin{dfn}
	Линия, заметаемая точечной частицей в ПВ, называется \emph{мировой линией}.
\end{dfn}
\begin{figure}[ht]
    \centering
    \incfig[0.8]{1}
    \caption{Световой конус}
    \label{fig:1}
\end{figure}
В данном вопросе стоит обсудить \emph{принцип причинности} (рис.~\ref{fig:1}).
Любое событие из верхней части светового конуса (например в мировой точке $a$ 
находится в абсолютном будущем по отношению к вершине конуса $o$. Т.\:е. в
любой СО это событие произойдет позже, чем то, что произойдёт в вершине конуса.
Любое же событие в нижней части светового конуса (например, событие в мировой
точке $b$ находится в абсолютном прошлом по отношению к вершине конуса. В
прошлом или в будущем событие вне светового конуса (например, событие в
мировой точке  $d$ по отношению к вершине конуса зависят от СО.

Объясним откуда следуют эти утверждения. Любая мировая точка внутри светового
конуса соединяется с его вершиной отрезком с интервалом $\Delta s^2=c^2 \Delta t^2- \Delta
t^2- \Delta \mathbf{x}^2 >0$, т.\:к. для такого интервала смещение в пространстве
и во времени связаны как $|\Delta \mathbf{x}|<c|\Delta t|$. Более того
$\Delta t$ не возможно положить равным нулю выбором СО,  т.\:к. иначе нарушилось
бы условие $\Delta s^2 >0$. Поэтому если $\Delta t>0$ в одной СО, то
$\Delta t >0$ и в любой другой СО. Тоже верно и в случае, если  $\Delta t<0$.
Интервалы, для которых верно  $\Delta s^2>0$ называются  \emph{времениподобными}.

Любая мировая точка вне светового конуса соединяется с его вершиной отрезком
с $\Delta s^2 <0$,  т.\:к. для таких интервалов $|\Delta \mathbf{x}|>c |\Delta t|$.
Поэтому, для таких интервалов выбором СО можно положить $\Delta =0$,  т.\:е.,
меняя систему отсчёта, можно сменить знак $\Delta t$. Следовательно, если
событие было в прошлом по отношению к вершине конуса в одной СО, то его можно
положить в будущее по отношению к вершине выбором другой СО. Интервалы, для
которых верно $\Delta s^2 <0$ называются  \emph{пространственноподобными}.

И наконец, любая точка на световом конусе соединяется с его вершиной интервалом
с $\Delta s^2 = 0$,  т.\:к. для такого интервала $|\Delta \mathbf{x}|=c
|\Delta t|$. Такие интервалы называются  \emph{нулевыми} или \emph{светоподобными}.
Очевидно, является ли интервал светоподобными, пространственноподобным или
временноподобным не зависит от СО, т.\:к. величина интервала не зависит от
СО.

В СТО дновременность относительна. См., например, вопрос \ref{subsec:3}
\subsection{Лоренцево сокращение длин \emph{(стр. 10)}}
\label{subsec:3}
Рассмотрим движение стержня длины $l_0$ вдоль своей оси симметрии со скоростью
$v$. Пусть в ИСО стержня его задний конец находится в начале координат  $x'_1=0$,
а передний, соответственно в  $x'_2=l_0$. Пусть теперь в ЛСО в какой-то момент
времени  $t$ (по часам ЛСО) задний конец стержня находится в точке $x_1$, а
передний в точке $x_2$. Найдём  $l=x_2-x_1$.
Из формулы для буста Лоренца (см. вопрос \ref{subsec:1}) мы знаем как $x'$ и
$t'$ связаны с  $x$ и  $t$:
\begin{gather*}
	x'_1\equiv 0 = (x_1-\beta c t)\gamma\\
	x'_2\equiv l_0 = (x_2-\beta c t)\gamma\\
	t'_1=\left( t-\frac{\beta x_1}{c} \right) \gamma\\
	t'_2=\left( t-\frac{\beta x_2}{c}\gamma \right) 
\end{gather*}
Из рассматриваемых уравнений следует, что
\[
	t'_2-t'_1=\frac{\gamma \beta}{c}(x_2-x_1)\equiv \frac{\gamma\beta l}{c} > 0
.\]
Т. е. если в ЛСО концы стержня находятся в точках $x_1$ и $x_2$ одновременно,
то в системе отсчёта стержня попадание его заднего конца в точку  $x_1$, а
переднего --- в точку  $x_2$ в ЛСО не есть одновременные события. Но при этом
интервалы в двух ИСО должны быть равны:
\begin{gather*}
	ds^2=c^2(t-t)^2-(x_2-x_1)^2=-(x_2-x_1)^2\equiv -l^2\\
	ds^2=c^2(t'_2-t'_1)^2-(x'_2-x'_1)^2=l^2\gamma^2\beta^2-l_0^2
.\end{gather*}
Поэтому $l^2=-l^2\gamma^2 \beta^2+l_0^2$, отсюда
\[
l=l_0 /\gamma
.\]
Это явление в СТО и называется \emph{Лоренцевым сокращением длин}.
\subsection{Релятивистское сложение скоростей вдоль одного и того же
направления \emph{(стр. 11)}}
Пусть $K$ --- ЛСО; $K_1$ --- ИСО, движущаяся с скоростью $v$ в ЛСО;
$K_2$ --- ИСО, движущаяся со скоростью $u$ в системе  $K_1$.
Сделаем два буста Лоренца подряд --- сначала для перехода из  $K$ в $K_1$,
а затем из  $K_1$ в  $K_2$. Т.\:е. мы должны применить композицию двух
Лоренцевских бустов с параметрами: $\cosh \alpha_1 = 1 /\sqrt{1-\frac{v^2}{c ^2}} $,
$\tanh \alpha_1=v /c$ и $\cosh  \alpha_2=1 /\sqrt{1- \frac{u^2}{c^2}} $,
$\tanh \alpha_2 =u /c$:
\[
\begin{pmatrix} ct_1 \\ x_1 \end{pmatrix} =
\begin{pmatrix} \cosh \alpha_1  & \sinh \alpha_1 \\
\sinh \alpha_1 & \cosh \alpha_1\end{pmatrix} 
\begin{pmatrix} ct \\ x \end{pmatrix}, \quad
\begin{pmatrix} ct_2 \\ x_2 \end{pmatrix} =
\begin{pmatrix} \cosh \alpha_2  & \sinh \alpha_2 \\
\sinh \alpha_1 & \cosh \alpha_2\end{pmatrix} 
\begin{pmatrix} ct_1 \\ x_1 \end{pmatrix}
.\]
В результате мы получим опять буст Лоренца для перехода из $K$ прямо в $K_2$,
а параметр его будет равен $\alpha=\alpha_1+\alpha_2$.
Т.\:е. скорость системы $K_2 $ по отношению к  системе $K$ будет равна
 \[
	 \frac{V}{c}=\tanh (\alpha_1+\alpha_2)=
	 \frac{\tanh \alpha_1 +\tanh \alpha_2}{1 +\tanh \alpha_1 \tanh \alpha_2}=
	 \frac{v /c+u /c}{1+ \frac{vu}{c^2}}
.\] 
\[
\frac{V}{c}=\frac{v /c +u /c}{1+\frac{vu}{c^2}}
.\] 
\subsection{Вычислить $\operatorname{div} \mathbf{r}$}
\[
	\operatorname{div} \mathbf{r}=\partial_i x_i=\delta_{ii}=3
.\] 
\subsection{Вектор 4-скорости и 4-ускорения и их скалярное произведение.}
Компоненты 4-вектора скорости имеют вид
\[
	u^\mu=\left( \gamma,\,\gamma \boldsymbol{\beta} \right) 
,\] 
где $\boldsymbol{\beta}=\mathbf{v} /c$.
4-ускорение определяется как:
\[
w^\mu\equiv \frac{du^\mu}{ds}= \frac{du^\mu}{cdt\sqrt{1-v^2 /c^2} }
.\]
\[
	u^\mu u_\mu=1 \quad \Rightarrow \quad \frac{d}{ds}\left( u^\mu u_\mu \right) =0
	\quad \Rightarrow \quad w^\mu u^\mu=0
.\] 
\subsection{Компоненты 4-импульса и связь энергии с трёхмерным импульсом.}
Компоненты ковариантного 4-импульса:
\[
	p^\mu=\left( mc\gamma,\,m\mathbf{v}\gamma \right) =\left( \mathcal{E} /c,
	\mathbf{p}\right) 
.\]
Связь энергии с трёхмерным импульсом:
\[
	\mathbf{p}=\frac{\mathcal{E}\mathbf{v}}{c^2}, \qquad
	\frac{\mathcal{E}^2}{c^2}-\mathbf{p}^2=m^2c^2
.\] 
\subsection{Преобразование Лоренца произвольного вектора при бусте вдоль оси
$x$.}
См. вопрос \ref{subsec:1}.
\subsection{Эффективная масса нескольких частиц.}
\begin{dfn}
	\emph{Эффективной массой} $n$ частиц называют величину
	\[
		M=\left( \sum_{i=1}^{n} p^\mu_i \right)^2 
	.\] 
\end{dfn}
\subsection{Калибровочные преобразования потенциалов.}
\label{subsec:10}
\begin{dfn}
	\emph{Калибровочными преобразованиями потенциалов}  $\mathbf{A}$ и
	$\phi$ называют такие преобразования данных потенциалов, при которых
	построенные ЭМ поля $\mathbf{E}$ и $\mathbf{B}$ по новым потенциалам
	не будут отличаться от ЭМ полей, построенных по старым потенциалам.
\end{dfn}
\subsection{Калибровочные преобразования потенциалов в трёхмерной форме \emph{(стр. 40)}}
\label{subsec:11}
\begin{align*}
	\mathbf{A}'&=\mathbf{A}+\grad \alpha,\\
	\phi'&=\phi-\frac{1}{c}\frac{\partial \alpha}{\partial t} 
,\end{align*}
где $\alpha=\alpha(t,\,\mathbf{x})$ --- произвольная дифференцируемая функция.
Прямой проверкой следует из вопросов \ref{subsec:10}, \ref{subsec:13}.
\subsection{Калибровочные преобразования потенциалов в четырехмерной форме
\emph{(стр. 41)}}
	См. вопрос \ref{subsec:11}.
	
\emph{Электромагнитный  потенциал} имеет следующий вид
\[
	A^\mu\equiv(\phi,\,\mathbf{A})\quad \Rightarrow\quad A_\mu=(\phi,\,-\mathbf{A})
.\]
Калибровочные преобразования для данного потенциала можно записать следующим
образом
\[
A'_\mu=A_\mu-\partial_\mu \alpha
.\] 
\subsection{Выражения для $\mathbf{E}$ и $\mathbf{B}$ через компоненты 4-потенциала
\emph{(стр. 40)}}
\label{subsec:13}
По определению
\[
\mathbf{B}=\rot \mathbf{A}
.\] 
\emph{Четвёртное уравнение Максвелла}
\begin{equation*}
	\rot \mathbf{E}=-\frac{1}{c}\frac{\partial \mathbf{B}}{\partial t} 
\end{equation*}
при данной подстановке принимает вид
\[
	\rot \left( \mathbf{E}+\frac{1}{c}\frac{\partial \mathbf{A}}{\partial t}  \right) =0
.\]
Полученное уравнение выполняется тождественно, если:
\[
\mathbf{E}+\frac{1}{c}\frac{\partial \mathbf{A}}{\partial t}=-\grad \phi
\quad \Rightarrow \quad \mathbf{E}=-\frac{1}{c}\frac{\partial \mathbf{A}}{\partial t}-
\grad \phi
,\]
для любого дифференцируемого поля $\phi(t,\,\mathbf{x})$, т.\:к. $\rot \grad \equiv 0$.
\subsection{Выражение для тензора электромагнитного поля через 4-вектор
потенциал.}
По определению
\[
F_{\mu\nu}=\partial_\mu A_\nu -\partial_\nu A_\mu
.\] 
\subsection{Сила Лоренца \emph{(стр. 44)}}
Уравение движения релятивистской частицы во внешнем ЭМ поле:
\[
	mc \frac{du^\mu}{ds}=\frac{e}{c}F^{\mu\nu}u_{\nu}
.\]
Пространнственные компоненты данного 4-мерного уравнения имееют вид:
\[
mc \frac{du^i}{dt}=\frac{e}{c}F^{i\nu} \frac{dz_\nu}{dt}\equiv
\frac{e}{c}\left( F^{i0} \frac{dz_0}{dt}+F^{ij}\frac{dz_j}{dt} \right) 
.\]
Чтобы получить это уравнение, мы умножили левую и правую части исходного
уравнения на $ds /dt$, тем  самым заменив  $du^\mu /ds$ на  $du^\mu/dt$, а
$u_\nu=dz_\nu /ds$ на  $dz_\nu /dt$.

Вспоминая, что $mcu^i=p^i$, $dz_0 /dt=c$ и выражая компоненты тензора $F^{0i}$ 
и $F^{ij}$ через компоненты ЭМ полей $B_i$ и $E_i$, получаем уравнение:
 \[
\frac{dp_i}{dt}=eE_i+\frac{e}{c}\epsilon_{ijk}v_jB_k
,\] 
которое в вектороной форме имеет вид
\[
\frac{d\mathbf{p}}{dt}=e \mathbf{E}+\frac{e}{c}\mathbf{v} \times
\mathbf{B} 
.\] 
\subsection{Скорость дрейфа в скрещенных электромагнитных полях \emph{(стр. 49)}}
Ограничимся рассмотрением нерелятивистского случая, т.\:е. когда
$v(t) \ll c,\; \forall t$. Тогда $\mathbf{p}\approx m \mathbf{v}$. Пусть поле
$\mathbf{B}$ направлено вдоль оси $z$, а плоскость, проходящая через  $\mathbf{E}$ и
$\mathbf{B}$, совпадает с $yz$. Тогда уравнения движения частицы
 \[
	 m \dot{\mathbf{v}}=e \mathbf{E}+\frac{e}{c}\left[ \mathbf{v}\times \mathbf{B} \right] 
\] 
запишутся в виде
\begin{gather*}
	m  \ddot{x}= \frac{e}{c} \dot{y} B,\\
	m \ddot{y} = e E_y- \frac{e}{c} \dot{x} B,\\
	m \ddot{z}=e E_z
.\end{gather*} 
Из последнего уравнения очевидно следует $z(t)= \frac{e E_z t^2}{2m}+v_{0z}t$ ---
обычное равноускоренное движение. Первые два уравнения в этой системе можно
записать как одно комплексное:
\[
	\frac{d}{dt}\left( \dot{x}+i \dot{y} \right) + i\omega \left( 
	\dot{x}+i \dot{y} \right) =i \frac{e}{m}E_y,
\] 
где $\omega = e B / mc$ --- частота Лармора. Решение полученного уравнения есть
сумма общего решения однородного уравенения  $a e ^{-i\omega t}$, с амплитудой
$a$ следующей из начальных условий, и частного решения неоднородного уравнения.
В качестве последнего мы выберем $\left( \dot{x} + i \dot{y} \right)_{par}=
e E_y / m \omega= c E_y /B$. Т.\:е. общее решение  рассмотриваемого уравнения
есть:
\[
	\dot{x}+i \dot{y}= a t ^{i \omega t} + \frac{c E_y}{B}
.\]
Выберем начальные условия такими, чтобы $a$ была действительной. Тогда:
 \[
	 \dot{x}= a \cos \omega t+ \frac{c  E_y}{B},\quad \dot{y}=-a \sin \omega t
.\] 
Полученные компоненты  сккороссти частицы являются периодическими функциями. Их
средние по времени значения равны:
\[
	\overline{\dot{x}}=\frac{c E_y}{B}, \quad \overline{\dot{y}}=0
.\]
и определяют среднюю скорость движения заряда в скрещенных ЭМ полях ---  скорость
\emph{электрического дрейфа}. Её направление перпендикулярно обоим полям и
не зависит от знака заряда. В векторном виде её можно записать как
\[
	\mathbf{v}_{dr}=\frac{c}{\mathbf{B}^2}\mathbf{E}\times\mathbf{B}
.\] 
\subsection{Магнитное зеркало \emph{(стр.54)}}
\marginnote{\textsf{Роман Солецкий}}[0cm]
Заряженная частица движется в магнитном поле по спирали с радиусом
$r=mv_{\bot}/eH$. Если поле переменное, то адиабатически сохраняется 
\[I=\oint p_{\bot} dq_{\bot}=2\pi r m v_{\bot}\sim \frac{v_{\bot}^2}{H}\]
При движении в магнитном поле энергия сохраняется, поэтому $v_{\bot}^2$
ограничено сверху и частица не сможет преодолеть барьер в некоторое $H_{\max}$
\subsection{Вычислить среднее $\left<\left(\vec{a},\,\vec{n}\right)
\left( \vec{b},\,\vec{n} \right) \right>$ по всем направлениям единичного
вектора $\vec{n}$ при постоянных  $\vec{a}$, $\vec{b}$}
\marginnote{\textsf{Роман Солецкий}}[0cm]
$\left< n_i n_i\right>$ не зависит от $i$. Из независимости компонент 
\[
\sum_{i=1}^3\left< n_i n_i\right>=\left<\sum_{i=1}^3 n_i n_i\right>=1
\quad \Rightarrow \quad \left< n_i n_i\right>=\frac{1}{3}.
\]
Отсюда 
\begin{gather*}
	\left< n_i n_k\right>=\frac{\delta_{ik}}{3},\\
	\left<\left(\vec{a},\,\vec{n}\right)
\left(\vec{b}
,\,\vec{n}\right)\right>=\left< a_i n_i b_k n_k\right>=\frac{a_i b_k \delta_{ik}}{3}=
\frac{\left(\vec{a},\,\vec{b}\right)}{3}
\end{gather*}
\subsection{Вычислить среднее $\left<\left[\vec{a},\,\vec{n}\right]
\left( \vec{b},\,\vec{n} \right) \right>$ по всем направлениям единичного
вектора $\vec{n}$ при постоянных  $\vec{a}$, $\vec{b}$}
\marginnote{\textsf{Роман Солецкий}}[0cm]
\[
	\left<[\vec{a},\,\vec{n}]\left(\vec{b},\vec{n}\right)\right>=\left<
	\epsilon_{ijk}
	a_j n_k b_l n_l\right>=\frac{\left[\vec{a},\,\vec{b}\right]}{3}
\]
\subsection{Четыре-вектор тока и его компоненты \emph{(стр. 60)}}
4-вектор плотности тока:
\[
	j^\mu=(\rho c,\,\mathbf{j})
.\] 
\marginnote{\textsf{Роман Солецкий}}[0cm]
$de=\rho dV$ - количество заряда в области (скаляр).
\[
de dx^i=\rho dV dt \frac{dx^i}{dt}
\]
$dV dt$ - скаляр, $de dx^i$ - 4-x вектор, $\rho dx^i/dt$ - 4-х вектор тока $j^i$.
$j^i=\left(c\rho,\,\rho\vec{v}\right)=\left(c\rho,\,\vec{j}\right)$
\subsection{Уравнение непрерывности в четырехмерной и трехмерной форме 
\emph{(стр. 60)}}
Из уравнения Максвелла
\[
	\partial_\mu F^{\mu\nu}=\frac{4\pi}{c}j^\nu
,\] 
применяя к обеим его сторонам 4-дивергенцию, получаем:
\[
\frac{4\pi}{c}\partial_\nu j^\nu=\partial_\nu \partial_\mu F^{\mu\nu}=
\partial_\nu \partial_\mu (\partial^{\mu} A^\nu-\partial^\nu A^\mu)=
\partial_\nu \partial^2 A^\nu-\partial_\mu \partial^2 A^\mu=0
,\] 
где мы обозначили $\partial^2=\partial_\alpha \partial^\alpha$. Таким образом,
из уравнений Максвелла следует \emph{уравнение непрерывности}
\marginnote{\textsf{Роман Солецкий}}[0cm]
\[\frac{\partial j^\mu}{\partial x^\mu}=0\]
\[
\frac{\partial \rho}{\partial t}+\operatorname{div} \mathbf{j}=0
\]
\subsection{Плотность энергии электромагнитного поля \emph{(стр. 75)}}
Распишем покомпонентно ТЭИ через ЭМ поля  $\mathbf{E}$ и
$\mathbf{B}$
\[
T_0^0=-\frac{1}{4\pi}\left[ F^{0\mu}F_{0\mu}-\frac{1}{4}\delta_0^0 F^2 \right]=
-\frac{1}{4\pi}\left[ F^{0i}F_{0i}-\frac{1}{2}\left( \mathbf{B}^2-\mathbf{E}^2
\right) \right] =\frac{1}{8\pi}\left( \mathbf{E}^2+\mathbf{B}^2 \right) 
.\] 
Величина $W\equiv T_0^0$ является  \emph{плотностью энергии ЭМ поля}.
Действительно, Лагранжева плотность ЭМ поля равна:
\marginnote{\textsf{плохо понел}}[0cm]
\[
\mathcal{L}\propto F_{\mu\nu}F^{\mu\nu}\propto\mathbf{E}^2-\mathbf{B}^2\equiv
T-U
,\]
где $T \propto E_i^2\propto\left(\partial_0A_i-\partial_iA_0\right)^2$ ---
это кинетическая энергия ЭМ поля,  т.\:к. содержит производные по времени
от канонических переменных $\partial_0 A_i$,  т.\:е. зависит от скорости
изменения этих величин; при этом $U \propto B_i^2 \propto \left( 
\epsilon_{ijk}\partial_j A_k\right) ^2$ не  одержит производных по времени, а
потому является потенциальной энергией
Т.\:к. лагранжева плотность равна $\mathcal{L}=T-U$, то величина $W=T+U\propto
\mathbf{E}^2+\mathbf{B}^2$ должна быть плотностью энергии. 

\marginnote{\textsf{Роман Солецкий}}[0cm]
\[
\frac{1}{c}\frac{\partial \mathbf{E}}{\partial t}\mathbf{E}+\frac{1}{c}\frac{\partial \mathbf{H}}{\partial t}\mathbf{H}=\mathbf{E}\left(\operatorname{rot}\mathbf{H}-\frac{4\pi}{c}\mathbf{j}\right)-\mathbf{H}\operatorname{rot}\mathbf{E}=-\frac{4\pi}{c}\mathbf{E}\mathbf{j}-\operatorname{div}{[\mathbf{E},\mathbf{H}]}
\]
\[
\frac{1}{c}\frac{\partial \mathbf{E}}{\partial t}\mathbf{E}+\frac{1}{c}\frac{\partial \mathbf{H}}{\partial t}\mathbf{H}=\frac{1}{2c}\frac{\partial (\mathbf{E}^2+\mathbf{H}^2)}{2}
\]
\[\int \mathbf{E}\mathbf{j} dV=\int \mathbf{E}\rho dV \mathbf{v}=\sum e\mathbf{E}\mathbf{v}=\frac{dE_{\text{кин}}}{dt}\]
\[\frac{d}{dt}\left[E_{\text{кин}}+\int dV\frac{\mathbf{E}^2+\mathbf{H}^2}{8\pi}\right]=-\int dV\frac{c}{4\pi}\operatorname{div}{[\mathbf{E},\mathbf{H}]}=-\oint dS \frac{c}{4\pi}[\mathbf{E},\mathbf{H}]\]
Слева записано изменение полной энергия системы заключенной в некоторой оболочке, а справа то что эта энергия изменяется при излучении из оболочки. Полная энергия электромагнитного поля 
\[W=\frac{\mathbf{E}^2+\mathbf{H}^2}{8\pi}\]
\subsection{Вектор Умова-Пойнтинга \emph{(стр. 75)}}
Рассмотрим 3-вектор:
\[
S_i=-cT_i^0=\frac{c}{4\pi}F^{0\mu}F_{i\mu}=\frac{c}{4\pi}F^{0j}F_{ij}=
\frac{c}{4\pi}(-E_j)(-\epsilon_{ijk}B_k)=\frac{c}{4\pi}\epsilon_{ijk}E_j B_k
,\]
где $T_i^{0}$--- вектор из компонент ТЭИ.
Величина
\[
\mathbf{S}=\frac{c}{4\pi}\left[ \mathbf{E}\times\mathbf{B} \right] 
\]
называется \emph{вектором Умова-Пойнтинга}.

\marginnote{\textsf{Роман Солецкий}}[0cm]
Из предыдущего билета \[\mathbf{S}=\frac{c}{4\pi}[\mathbf{E},\mathbf{H}]\]
\subsection{Вектор потенциал $A$ для плоской и монохроматической
электромагнитной
волны.}
hi
\subsection{Поляризация плоской монохроматической электромагнитной волны.}
hi
\subsection{Векторы $E$, $B$ и Умова-Пойнтинга в плоской и монохроматической
электромагнитной волне.}
hi
\subsection{Классический радиус электрона и как он возникает в выражениях,
описывающих рассеяние электромагнитных волн.}
hi
\subsection{Аберрация света.}
hi
\subsection{Собственное время.}
hi
\subsection{Вычислить $\operatorname{grad} \frac{1}{\left| \vec{r} \right| }$}
hi
\subsection{Вычислить $\operatorname{grad}
	\frac{1}{\left( \vec{k},\vec{r}\right)}
$, где $\vec{k}$ --- постоянный вектор.}
hi
\subsection{Вычислить $\operatorname{grad} e ^{i\left( \vec{k},\,\vec{r}
\right) }$, где $\vec{k}$ --- постоянный вектор.}
hi
\subsection{Вычислить $\epsilon_{ijk}x_i x_k$.}
hi
\subsection{Вычислить $\delta_{ij}\partial_i x_k$.}
hi
\subsection{Вычислить $\delta_{ij}\partial_i x_j$.}
hi
\subsection{Действие для свободной релятивистской частицы.}
hi
\subsection{Вывести формулу для эффекта Доплера.}
hi
\subsection{Может ли свободный электрон излучить фотон? Объяснение.}
hi
\subsection{Действие для релятивистской частицы во внешнем электромагнитном
поле.}
hi
\subsection{Уравнение движения для релятивистской частицы во внешнем
электромагнитном поле в 4-мерной форме.}
hi
\subsection{Обобщенный импульс и энергия.}
hi
\subsection{Получите инварианты поля в четырехмерной (через тензор поля)
исходя и тензора э-м поля.}
hi
\subsection{Инварианты электромагнитного поля в трехмерной форме (через $E$ и
$B$).}
hi
\subsection{Вычислить среднее $\left<\left[ \vec{a},\,\vec{r} \right] \vec{r}
\right>$ по всем направлениям вектора $\vec{r}$ при постоянных  $\left| 
\vec{r}\right| $, $\vec{a}$, $\vec{b}$.}
hi
\subsection{Вычислить среднее $\left<\left[ \vec{a},\,\vec{n} \right]
\left[ \vec{b},\,\vec{n} \right]
\right>$ по всем направлениям вектора $\vec{n}$ при постоянных 
$\vec{a}$, $\vec{b}$.}
hi
\subsection{Первая и вторая пара уравнений Максвелла в четырехмерной форме.}
hi
\subsection{Дипольный электрический момент и поле, создаваемое им.}
hi
\subsection{Квадрупольный момент.}
hi
\subsection{Энергия электрического диполя и квадруполя во внешнем поле.}
hi
\subsection{Потенциальная энергия взаимодействия диполя с диполем.}
hi
\subsection{Закон Био-Савара – магнитное поле, создаваемое стационарным током.}
hi
\subsection{Калибровка Лоренца и вторая пара уравнений Максвелла в ней.}
hi
\subsection{Калибровка Кулона и уравнение на три-вектор потенциал $A$ в
присутствии стационарного тока.}
hi
\subsection{Дипольный магнитный момент и поле, создаваемое им.}
hi
\subsection{Прецессия магнитного момента в магнитном поле. Частота Лармора.}
hi
\subsection{Запаздывающие потенциалы.}
hi
\subsection{Получить Потенциалы Лиенара-Вихерта в трехмерной и четырехмерной
форме из запаздывающих потенциалов.}
hi
\subsection{Волновая зона. Характер поведения полей $E$ и $B$ вблизи
двигающегося заряда.}
hi
\subsection{Длина формирования излучения или длина когерентности.}
hi
\subsection{Характер распределения по углам излучения в ультрарелятивистском
случае.}
hi
\subsection{Интенсивность излучения в дипольном приближении.}
hi
\subsection{Характерная частота при синхротронном излучении.}
hi
\subsection{Радиационная сила трения. Критерий применимости.}
hi
\subsection{Лоренцева линия. Естественная ширина линии.}
hi
\subsection{Формула Томсона для сечения рассеяния.}
hi
\subsection{Тензор электромагнитного поля и связь его компонент с $E$ и $B$.}
hi
\subsection{Гамильтониан частицы в нерелятивистском приближении во внешнем
электромагнитном поле.}
hi
\subsection{Можно ли превысить скорость света при движении под действием
постоянной силы? Объяснение.}
hi
\subsection{Четыре-вектор тока для точечной частицы.}
hi
\subsection{Первая и вторая пара уравнений Максвелла в трехмерной форме.}
hi
\subsection{Тензор энергии-импульса для точечной частицы.}
hi
\subsection{Тензор энергии-импульса электромагнитного поля.}
hi
\subsection{Закон сохранения тензора энергии-импульса.}
hi
\subsection{Уравнение Пуассона и его решение. Потенциал Кулона.}
hi
\subsection{Разложение электромагнитного поля на осцилляторы. Фурье разложение 
$A$, $E$ и $B$.}
hi
\subsection{Действие для осцилляторов (собственных колебаний)
электромагнитного поля.}
hi
\subsection{Запаздывающая функция Грина для электромагнитного поля и ее
свойства.}
hi
\subsection{Получить запаздывающие потенциалы из запаздывающей функции Грина.}
hi
\subsection{Характер зависимости поля произвольно двигающегося заряда от
расстояния. Сколько слагаемых в $E$ и $B$? Как они падают с расстоянием?
Как зависят от ускорения?}
hi
\subsection{Мощность потерь на излучение в релятивистском случае и его связь
с полной интенсивностью излучения.}
hi
\subsection{Критерий применимости силы радиацонного терия.}
hi
\subsection{Критерий применимости нерелятивистского приближения для излучения.}
hi
\subsection{Критерий применимости формулы Томсона для рассеяния.}
hi
\end{document} % конец документа
